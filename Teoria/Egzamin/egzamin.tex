\documentclass[11pt]{article}
\usepackage{polski}
\usepackage[utf8]{inputenc}
\usepackage[T1]{fontenc}
\usepackage{graphicx}
\usepackage{grffile}
\usepackage{longtable}
\usepackage{hyperref}
\usepackage{wrapfig}
\usepackage{rotating}
\usepackage[normalem]{ulem}
\usepackage{amsmath}
\usepackage{textcomp}
\usepackage{amssymb}
\usepackage{capt-of}
\usepackage{mathtools}
\usepackage{geometry}
\usepackage{tabto}

\newcommand{\norm}[1]{\left\lVert#1\right\rVert}


 \geometry{
 a4paper,
 total={170mm,257mm},
 left=20mm,
 top=20mm,
 }

 \hypersetup{
    colorlinks=true,
    linkcolor=blue,
    filecolor=magenta,
    urlcolor=cyan,
}

\title{\textbf{Metody numeryczne}}
\author{Wojciech Kubiak}
\date{\today}

\begin{document}
\maketitle

\section{Arytmetyka zmiennopozycyjna, standard IEEE}
\begin{itemize}
    \item \textbf{Liczby maszynowe: }liczby rzeczywiste które można reprezentować w komputerze.
    \item \textbf{Reprezentacja liczby zmiennoprzecinkowej: }w arytmetyce podwójnej precyzji jest to ciąg 64 bitów, w pojedyńczej jest to ciąg 32 bitów.
    \item \textbf{Epsilon: }jest to najmniejsza dodatnia liczba spełniająca równanie $1 + \mathcal{E} \neq 1$
    \item \textbf{Wykładnik po przesunięciu "shift": }Jest to liczba zapisana na 8 bitach może być z zakresu -126 do 127. Używa się go do kodowania z nadmiarem.
\end{itemize}

\section{Nadmiar i niedomiar}
\begin{itemize}
    \item \textbf{Nadmiar: }jeżeli w czasie obliczeń liczba $x$ (np. wynik operacji arytmetycznych) znajdzie się poza dopuszczalnym zakresem liczb. Kończy działanie programu.
    \item \textbf{Niedomiar: }automatycznie zapamiętywana jest liczba $x$ jako zero bez przerwania działania programu.
\end{itemize}

\section{Błąd względny, bezwzględny, zaokrąglenie do najbliższej liczby parzystej}
\begin{itemize}
    \item \textbf{Błąd względny: }\quad $|\frac{a - \tilde{a}}{a}|$
    \item \textbf{Błąd bezwzględny: }\quad $|a-\tilde{a}|$
    \item \textbf{Metoda zaokrąglenia do najbliższej pażystej: }redukuje błąd całkowity obliczeń z uwagi na statycznie równą liczbę zaokrągleń w góre i w dół.
\end{itemize}

\section{Źródła błędów}
\begin{itemize}
    \item \textbf{zaokrąglenia} (związane z pracą w arytmetyce o skończonej pozycji)
    \item \textbf{obcięcia} (obliczeń do skończonej liczby kroków)
    \item \textbf{niepewność danych} (pojawiająca się w przypadku pracy na danych związanych z problemiami praktycznymi np. pomiarami fizycznymi)
\end{itemize}

\section{Uwarunkowanie zadania}
\begin{itemize}
    \item \textbf{Definicja: }Jeżeli niewielka zmiana danych wejściowych powoduje duże błędy w wyniku mówimy że nasze zadanie jest źle uwarunkowane. Wielkość charakteryzująca wpływ tych zaburzeń nazywamy wskaźnikiem uwarunkowania zadania (ang. condition numbers). 
    \item \textbf{Dla funkcji wielu zmiennych} $cond(f(x,y,\dots)) = \frac{\partial f}{\partial x} \cdot \frac{x}{f(x,y)} + \frac{\partial f}{\partial y} \cdot \frac{y}{f(x,y)}\dots$
    \item \textbf{Dla iloczynu skalarnego wektorów} $cond(x,y) = \frac{\langle |x|, |y| \rangle}{|\langle x, y \rangle|}$
    \item \textbf{Dla macierzy} $cond(A) = \norm{A} \cdot \norm{A^{-1}}$
\end{itemize}

\section{Stabilność numeryczna algorytmu}
\begin{itemize}
    \item \textbf{Definicja: }algorytm jest niestabilny numerycznie jeżeli wprowadza duże błędy w dobrze uwarunkowanym zadaniu.
    \item \textbf{Jak stworzyć algorytm stabilny numerycznie ?}
    \begin{enumerate}
        \item Unikaj odejmowania wielkości obarczonych błędem (o ile to możliwe).
        \item Minimalizuj rozmiar wyników pośrednich względem wielkości rozwiązania.
        \item Upewnij się, że metody obliczeń są równoważne numerycznie (nie tylko matematycznie).
        \item Przedstawiaj aktualne wyrażenie jako \textbf{nowa wartość = poprzednia wartość + mała korekta} jeżeli mała korekta może być obliczona z dużą liczbą cyfr znaczących.
    \end{enumerate}
\end{itemize}
\section{Ilorazy różnicowe}
\begin{itemize}
    \item \textbf{Iloraz różnicowy rzędu k: }\quad $f[x_0,x_1,\dots,x_k] = \sum_{k=0}^i \bigg[\frac{f(x_k)}{\prod_{j=0,j \neg k}^i(x_k - x_j)}\bigg]$
    \item \textbf{Twierdzenie: }Wartość ilorazu różnicowego $f[x_0,x_1,\dots,x_k]$ nie zmienia się bezwzględu na permutacje argumentów $x_0,x_1,\dots,x_k$.
\end{itemize}

\section{Zalety postaci Lagrange’a i Newtona wielomianu interpolacyjnego}
\begin{itemize}
    \item \textbf{Zalety Newtona: } jest możliwość dodania nowego punktu bez zmiany wcześniej obliczonych wartości
\end{itemize}
\end{document}